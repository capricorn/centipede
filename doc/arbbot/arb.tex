\documentclass[12pt]{article}
\usepackage[utf8]{inputenc}
\usepackage{amsmath}

\title{pi arb strats}
\author{cap}

\begin{document}
\maketitle

The book is modelled like so:
\begin{align*}
\text{bids} &= \{ (b_1, q_1), ..., (b_n, q_n) \}\\
\text{asks} &= \{ (a_1, q_1), ..., (a_n, q_n) \}
\end{align*}

$bids$ and $asks$ are totally ordered over $>$, and 
\begin{align*}
    b_i - b_{i+1} &\geq 1\\
    a_i - a_{i+1} &\geq 1
\end{align*}

For a book with $n$ price levels, the top of the book is at index $1$, and the bottom of the book at $n$.
The total quantity of shares up to and including a price level $k$ for bids and asks is
\begin{align*}
Q_b(k) &= \sum_{i=1}^{k} q_i\\
Q_a(k) &= \sum_{i=1}^{k} q_i
\end{align*}

\section*{Current Strategy}

For the time being, this strategy has constants for deciding what quantities to purchase, and what
levels to sell or liquidate at. Hopefully this is improved in the future to account for other factors.

\subsection*{Placing bids}

% Our notation could probably be improved here
First, find the price level $k$ such that
\begin{equation}
    Q_b(k) \leq 100 \wedge (Q_b(k) \geq (Q_b(i) \leq 100))
\end{equation}

Now the bids we're interested in are
$$
B = \{ (b_1, q_1), ..., (b_k, q_k) \}
$$

If $ B = \emptyset $, then the only option is to place a bid at $b_1 + 1$ (if an arb opportunity exists).
Otherwise, examine the bottom index $k$. If $b_k - b_{k+1} > 1$, then place the order at $b_{k+1}+1$.

\subsection*{Adjusting bid}

Everytime an orderbook change occurs, the arb position needs re-evaluated. If the arb opportunity is lost,
cancel the order and liquidate the remainder at $b_1$. Optionally, adjust our sell price if more profit
can be made.
Otherwise, calculate $Q_k$ again. If $k$ changed,
verify an arb opportunity exists, and adjust the order. Hold any shares that were already purchased.

\subsection*{Placing asks}

\subsection*{Adjusting asks}

\subsection*{Arb opportunities}

An arb opportunity exists if:
\begin{align*}
a_1 - b_k &== 1 \wedge aq_1 \leq 25\\
a_1 - b_k &> 1\\
\end{align*}

\end{document}
